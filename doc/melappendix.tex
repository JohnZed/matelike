
\subsection*{elLoad}

You MUST call \texttt{elLoad} before using any other \melike
functions.

\begin{lstlisting}[commentstyle=\ttfamily]
function elLoad(withIpopt, forceReload)
% Loads ipopt, matelike, and intlab libraries
%
% USAGE:
%     elLoad
%  or:
%     elLoad(withIpopt, forceReload)
%
% If you use the second form and 'withIpopt' is true, then the Ipopt solver
% will be loaded. If you call elLoad twice, the second call will do nothing,
% unless the 'forceReload' option is set to true. The first form, with no
% arguments, will work fine for most users of the default zipsolver method.
% 
% IMPORTANT: You will need to edit BASEPATH below to point to the directory
% containing the 'matelike'  and 'zipsolver' folders on your system.
%
\end{lstlisting}

\linesect{elSetup}
\begin{lstlisting}[commentstyle=\ttfamily]
% Initializes an EL (or other Cressie-Read) problem
%
% Usage:
%
%   elike = elSetup(nObs, nMom, nTheta, momFunction, ...)
%
% Inputs:
%     'nObs', 'nMom', 'nTheta' are the number of observations, number of
%          moment conditions, and number of elements in theta
%
%     Extra arguments may follow 'momFunction' to set various options
%         for the optimization process. Options are specified as pairs,
%         where the first argument is the option name and the second
%         is the option value, like:
%
%             elSetup(nobs,nmom,ntheta,func, 'verbose', false)
%
%         which sets the 'verbose' option to false, greatly reducing the
%         output displayed.
%
%     'momFunction' is a function handle to compute the moment
%         conditions for each observation, with the form:
%      
%        M = momFunction(theta, elike, mnum, newTheta)
%
%          'mnum' indicates which moment to evaluate, 'theta' is the point
%          at which to evaluate it, and 'elike' is the same structure
%          that was returned from elSetup. If you need to pass additional
%          paraneters to your function, you can stash them in 'elike' as
%          fields.
%
%          'newTheta' is true if this theta differs from the one
%          that was passed in the most recent call. If you need to
%          precompute some values only once for each theta, 'newTheta'
%          tells you when it's time to redo that computation.
%
%          The returned value M should be an nObs x 1 vector containing
%          moment condition 'mnum' evaluated at each observation.
%          
%     *** IMPORTANT restrictions on momFunction ***
%
%          elSolve will use automatic differentiation (AD) to compute the
%          derivatives of momFunction. For smooth functions computed
%          analytically, this should work transparently. It will NOT work if
%          momFunction uses some iterative algorithm internally, is
%          non-differentiable, or uses some obscure external functions that
%          are not understood by the differentiation library (Intlab).
%
%          The matrix returned by momFunction must be of the special
%          AD type. This will happen by automatically if the result is
%          produced by operations involving 'theta', but it will fail
%          if you manually initialize M with something like 'M =
%          zeros(nObs,1)'. If you find this initialization convenient,
%          you can use: 'M = zeros(nObs,1) * theta(1)', which will
%          ensure that M has the correct type.
%
%          If automatic differentiation is not suitable for your model, you
%          can provide functions in 'elike.userCompJac' and
%          'elike.userCompHessTheta' to compute them. (See elLinearIV for
%          an example.)
%
% Output:
%
%     'elike' is a structure that can be passed to elSolve
%
% Requirements:
%
%     You must have Intlab installed to use elSetup. You also need
%     to call 'elLoad' sometime before elSetup to initialize the
%     matelike library.
%
\end{lstlisting}

\linesect{elSolve}

\begin{lstlisting}[commentstyle=\ttfamily]
function res = elSolve(elike, meth, thetaGuess, pguess, varargin)
% Solves the empirical likelihood (or other CR-family) model in elike
%
% Primary usage:
%
%    res = elSolve(elike, method, thetaGuess, pguess, ...)
%
% Inputs:
% 
%   'elike' is a structure containing the problem definition. It must
%           be created by:  elike = elSetup(...);  See 'help elSetup'
%
%   'method' specifies which objective function to use. It can be 'EL'
%           for empirical likelihood, 'ET' for exponential tilting, or a
%           numerical value indicating the Cressie-Read lambda parmeter to
%           use. If 'method' is 'GMM', a simple two-stage GMM estimator
%           will be used (ignoring pguess and other EL-specific params).
%            
%   'thetaGuess' is the starting value for the theta parameters.
%
%   'pguess' is an optional starting value for the pi parameters. If this
%            argument is empty ([]) or omitted, pi=1/N will be used.
%
%
% Outputs:
%
%   'res' is a structure containing:
%
%      'res.theta'   -- estimated value for theta
%      'res.lagmult' -- lagrange multipliers on moment constraints
%      'res.p'       -- estimated pi for each observation
%      'res.numiter' -- number of iterations taken
%      'res.fval'    -- value of objective function at optimum
%
%
% Details:
%
%   elSolve will use the 'zipsolver' matlab package by default. If you
%   have Ipopt and its matlab interface installed, set 'elike.useIpopt = true'
%   to use it (Ipopt is more robust and, for large problems, faster than
%   zipsolver).
%
%   See 'help elSetup' for more info on configuring EL problems.
%   
\end{lstlisting}

\linesect{elLinearIV}

\begin{lstlisting}[commentstyle=\ttfamily]
function res = elLinearIV(X, Z, y, elMethod, varargin)
% Solves a linear instrumental variables problem with EL
%
% Usage:
%
%   res = elLinearIV(X, Z, y, elMethod)
%
%   'X' is a matrix containing the structural variables
%   'Z' is a matrix containing the instruments
%   'y' is a vector of the dependent variable
%   'elMethod' specifies the method to use ('EL','ET', or a Cressie-Read lambda
%       value). You can omit this argument to use the default 'EL'
%
%  The returned structure 'res' is the same as that returned by elSolve. The
%  field 'res.theta' contains the estimates for the coefficients on X. Note that
%  it contains a field 'elike' with the problem setup.  This may be useful to
%  pass to elEval(...), etc.
%
\end{lstlisting}

\linesect{elModelSumm}
\begin{lstlisting}[commentstyle=\ttfamily]
function elModelSumm(res, names)
% Print some basic information about the EL model solution 'res'
%
% Usage:
%
%   elModelSumm(res, names)
%
%   'res' is a solution object returned by elSolve(...)
%   'names' is an optional vector of names for the elements of theta
%
% Details:
%
%    Reports the estimated theta from the model, standard errors,
%    and a likelihood ratio-based test for the validity of the
%    overidentifying restrictions.
%
%    The standard errors are based on the normal asymptotic approximation
%    to the estimator of theta, using a variance matrix based on the moments
%    weighted by the computed 'pi' values.
%
\end{lstlisting}

Sample usage and output from \texttt{elModelSumm}:

\begin{verbatim}

>> names = strvcat('theta1', 'theta2');
>> elModelSumm(res, names);

nObs:           1000
Moments:        10
Theta #elems:   2
Method:         EL

Degree of OID:    8
Likelihood unres/restricted:     -6907.75528 /  -6910.03657
LR stat:           4.56258
Chi-square q:      0.19686
Using empirical likelihood

Estimated theta (and asymptotic stderr):

              theta1       1.03688      ( 0.02451)
              theta2       2.10858      ( 0.15262)

\end{verbatim}

\linesect{elLRTest}

See \texttt{matelike/demos/runSimpleDemo.m} for an example.

\begin{lstlisting}[commentstyle=\ttfamily]
function [pval,lrstat] = elLRTest(elres, R, b, joint)
% Computes an empirical likelihood ratio test of the hypotheses 'R*theta = b'
%
% Usage:
%
%    testres = elLRTest(elres)
% 
%    testres = elLRTest(elres, R, b, joint)
%
% Inputs:
%
%    'elres' is a result structure returned by elSolve
%
%    'R' is a K x nTheta matrix of linear hypotheses to test, while 'b' is a
%        K x 1 vector of values to test R against. If 'joint' is
%        omitted or false, then each row of the matrix is a separate
%        linear hypothesis to be tested of the form:
%
%            H0:  R(k,:) * theta = b
%
%        If 'joint' is true, then the vector hypothesis: H0: hyp * theta = b
%        is tested.
%
%        If 'R' is omitted, by default elLRTest tests separately that each
%        element of theta = 0. (So it is equivalent to 'hyp = eye(nTheta)')
%
% Outputs:
%
%    'pval' is K x 1 vector for separate estimation or a scalar for joint
%        estimation. It is the p-value for the LR test: the probability under
%        the null that the LR statistic would be at least as large as the one
%        obtained. It comes from the CDF of a chi-squared(dof) distribution,
%        where 'dof' is the rank of the hypothesis. If 'joint' is false,
%        separate p-values are computed for each hypothesis.
%
%    'lrstat' the LR statistic obtained from each hypothesis:
%
%        LR = -2 * ( L(theta_restricted) - L(theta_unres) )
%
% Notes:
%
%    Each test requires re-solving the model, so it may take some time for a
%    large model.
%
\end{lstlisting}

\linesect{elEval}

\begin{lstlisting}[commentstyle=\ttfamily]
function res = elEval(elike, method, theta, p, lambda)
% Evaluates the empirical likelihood function at the given point
%
% Usage:
%    res = elEval(elike, theta, p, lambda);
%
% 'elike' should be a structure obtained from elSetup(...)
%
% 'method' is 'EL','ET', or a CR constant parameter
%
% 'lambda' contains the Lagrange multipliers, but may be omitted
%          if you don't want the Hessian.
%
% 'res' contains the following fields:
%    'res.f'      --  objective function
%    'res.fgrad'  --  gradient of objective
%    'res.H'      --  Hessian of the Lagrangian (empty if lambda missing)
%    'res.c'      --  constraints ( sum(pi(i) * m(i) )
%    'res.cJ'     --  Jacobian of moments per observation
%    'res.mom'    --  Moment conditions for each observation:
%                       an (nObs x nMom) matrix with one column for each
%                       moment condition and one row for each observation.
%
\end{lstlisting}

\linesect{linearIV}

\begin{lstlisting}[commentstyle=\ttfamily]
function res = linearIV(X, Z, y, meth, endogVars)
% Solves an instrumental variables problem using linear methods
%
% Can use either two-stage least squares (TSLS) or limited-information
% maximum likelihood (LIML). OLS is also an option, though it ignores Z
% may be inconsistent.
%
% Usage:
%
%   res = linearIV(X, Z, y, method, endogVars)
%
%   'X' is a matrix containing the structural variables
%
%   'Z' is a matrix containing the instruments. If some structural
%       variables are exogenous and to be used as instruments, the should
%       appear in Z as well as X.
%
%   'y' is a vector of the dependent variable
%
%   'method' is 'TSLS', 'LIML', or 'OLS'
%       (OLS is included for comparison, despite being inconsistent)
%
%   'endogVars' is a vector with length equal to the number of structural
%       variables. An entry contains 1 (or true) if the corresponding
%       column of X represents an endogenous variable.
%       endogVars is only used for LIML -- it may be omitted in other cases
%
% The returned structure 'res' contains:
%
%    'res.theta'   -- estimated coefficients on 'X'
%    'res.k'       -- in the LIML case, the value of the k parameter
%                     computed for the k-class estimator. Otherwise
%                     left blank.
%
\end{lstlisting}

\linesect{gmmSolve}
\begin{lstlisting}[commentstyle=\ttfamily]
function res = gmmSolve(elike, thetaGuess, varargin)
% Solves for theta using 2-step GMM.
%
% Usage:
%
%   res = gmmSolve(elike, thetaGuess, W1, ...)
%
% Inputs:
%
%   'elike' is a structure containing the problem definition. It must
%           be created by:  elike = elSetup(...);  See 'help elSetup'
%
%   'thetaGuess' is an optional starting guess for theta.
%
%   'W1' is an optional first-stage weighting matrix. The identity
%           matrix is used if W1 is not provided.
%
% Outputs:
%
%   'res' is a structure like that returned by elSolve(...), containing
%         'res.theta' with the estimated value of theta and 'res.V'
%         with the estimated variance of theta.
%    
% Details:
%
%    This is not a very sophisticated GMM implementation and is included only
%    for simple comparisons. The second-step weighting matrix is
%    stored in 'res.elike.W'.
%
\end{lstlisting}
