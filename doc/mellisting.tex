\linesect{elLoad}
\begin{lstlisting}
  Loads ipopt, matelike, and intlab libraries
 \end{lstlisting}
{\small\textbf{Usage:}}
\begin{lstlisting}
 
     "\texttt{elLoad}"
  or:
      "\texttt{elLoad(solver, forceReload)}"
 \end{lstlisting}
{\small\textbf{Inputs:}}
\begin{lstlisting}
 
     "\texttt{solver}" is an optional string specifying the nonlinear solver to use.
 
     Valid solver options are:
 
         'zipsolver' -- matElike's default, built-in solver
         'ipopt' -- the Ipopt solver, which require the Ipopt MEX interface and
                 library to be installed
         'fmincon' -- the fmincon solver from Matlab's optimization
                 toolbox. This is only available with optimization toolbox
                 version 4.0 (Matlab R2008a) or greater, as the older versions
                 lack support for analytic Hessians with inequality constraints.
 
      In general, Ipopt is the most robust solver, but 'zipsolver' works well
      for most problems as long as they're reasonably well scaled. 'fmincon'
      appears to be less reliable, but slightly faster than zipsolver.
  
  IMPORTANT: You will need to edit BASEPATH below to point to the directory
  containing the 'matelike'  and 'zipsolver' folders on your system.
 

\end{lstlisting}
\linesect{elSetup}
\begin{lstlisting}
  Initializes an EL (or other Cressie-Read) problem
 \end{lstlisting}
{\small\textbf{Usage:}}
\begin{lstlisting}
 
    "\texttt{elike = elSetup(nObs, nMom, nTheta, momFunction, ...)}"
 \end{lstlisting}
{\small\textbf{Inputs:}}
\begin{lstlisting}
      "\texttt{nObs, nMom, nTheta}" are the number of observations, number of
          moment conditions, and number of elements in theta
 
      Extra arguments may follow 'momFunction' to set various options
          for the optimization process. Options are specified as pairs,
          where the first argument is the option name and the second
          is the option value, like:
 
              "\texttt{elSetup(nobs,nmom,ntheta,func, 'verbose', false)}"
 
          which sets the 'verbose' option to false, greatly reducing the
          output displayed.
 
      "\texttt{momFunction}" is a function handle to compute the moment
          conditions for each observation, with the form:
       
         "\texttt{M = momFunction(theta, elike, mnum, newTheta)}"
 
           "\texttt{mnum}" indicates which moment to evaluate, "\texttt{theta}" is the point
           at which to evaluate it, and "\texttt{elike}" is the same structure
           that was returned from elSetup. If you need to pass additional
           paraneters to your function, you can stash them in "\texttt{elike}" as
           fields.
 
           "\texttt{newTheta}" is true if this theta differs from the one
           that was passed in the most recent call. If you need to
           precompute some values only once for each theta, "\texttt{newTheta}"
           tells you when it's time to redo that computation.
 
           The returned value M should be an nObs x 1 vector containing
           moment condition "\texttt{mnum}" evaluated at each observation.
           
      *** IMPORTANT restrictions on momFunction ***
 
           elSolve will use automatic differentiation to compute the
           derivatives of momFunction. For smooth functions computed
           analytically, this should work transparently. It will NOT work if
           momFunction uses some iterative algorithm internally, is
           non-differentiable, or uses some obscure external functions that
           are not understood by the differentiation library (Intlab).
 
           The matrix returned by momFunction must be of the special AD
           type. This will happen by automatically if the result is produced by
           operations involving "\texttt{theta}", but it will fail if you manually
           initialize M with something like "\texttt{M = zeros(nObs,1)}".  If you find
           this initialization convenient, you can use:
              "\texttt{M = zeros(nObs,1) * theta(1)}",
           which will ensure that M has the correct type.
 
           If automatic differentiation is not suitable for your model, you
           can provide functions in "\texttt{elike.userCompJac}" and
           "\texttt{elike.userCompHessTheta}" to compute them. (See elLinearIV for
           an example.)
 \end{lstlisting}
{\small\textbf{Output:}}
\begin{lstlisting}
 
      "\texttt{elike}" is a structure that can be passed to elSolve
 \end{lstlisting}
{\small\textbf{Requirements:}}
\begin{lstlisting}
 
      You must have Intlab installed to use elSetup. You also need
      to call "\texttt{elLoad}" sometime before elSetup to initialize the
      matelike library.
 
  Author:  John Zedlewski (jzedlewski@hbs.edu)
 

\end{lstlisting}
\linesect{elSolve}
\begin{lstlisting}
  Solves the empirical likelihood (or other CR-family) model in elike
 \end{lstlisting}
{\small\textbf{Primary Usage:}}
\begin{lstlisting}
 
     "\texttt{res = elSolve(elike, method, thetaGuess, pguess, ...)}"
 \end{lstlisting}
{\small\textbf{Inputs:}}
\begin{lstlisting}
  
    "\texttt{elike}" is a structure containing the problem definition. It must
            be created by:  elike = elSetup(...);  See "\texttt{help elSetup}"
 
    "\texttt{method}" specifies which objective function to use. It can be 'EL'
            for empirical likelihood, 'ET' for exponential tilting, or a
            numerical value indicating the Cressie-Read lambda parmeter to
            use. If 'method' is 'GMM', a simple two-stage GMM estimator
            will be used (ignoring pguess and other EL-specific params).
             
    "\texttt{thetaGuess}" is the starting value for the theta parameters.
 
    "\texttt{pguess}" is an optional starting value for the pi parameters. If this
             argument is empty ([]) or omitted, pi=1/N will be used.
 
 \end{lstlisting}
{\small\textbf{Outputs:}}
\begin{lstlisting}
 
    "\texttt{res}" is a structure containing:
 
       "\texttt{res.theta}"   -- estimated value for theta
       "\texttt{res.lagmult}" -- lagrange multipliers on moment constraints
       "\texttt{res.p}"       -- estimated pi for each observation
       "\texttt{res.numiter}" -- number of iterations taken
       "\texttt{res.fval}"    -- value of objective function at optimum
       "\texttt{res.flag}"    -- set to 0 if optimization succceeded, nonzero otherwise
 \end{lstlisting}
{\small\textbf{Details:}}
\begin{lstlisting}
 
    elSolve will use the 'zipsolver' matlab package by default. If you have
    Ipopt and its matlab interface installed, set "\texttt{elike.solver = 'ipopt'}" to
    use it (Ipopt is more robust and, for large problems, faster than
    zipsolver).
 
    See 'help elSetup' for more info on configuring EL problems.
    
  Author: John Zedlewski (jzedlewski@hbs.edu)  
  

\end{lstlisting}
\linesect{elLinearIV}
\begin{lstlisting}
  Solves a linear instrumental variables problem with EL
 \end{lstlisting}
{\small\textbf{Usage:}}
\begin{lstlisting}
 
    "\texttt{res = elLinearIV(X, Z, y, elMethod)}"
 \end{lstlisting}
{\small\textbf{Inputs:}}
\begin{lstlisting}
 
    "\texttt{X}" is a matrix containing the structural variables
    "\texttt{Z}" is a matrix containing the instruments
    "\texttt{y}" is a vector of the dependent variable
    "\texttt{elMethod}" specifies the method to use ('EL','ET', or a Cressie-Read lambda
        value). You can omit this argument to use the default 'EL'
 
   The returned structure "\texttt{res}" is the same as that returned by elSolve. The
   field "\texttt{res.theta}" contains the estimates for the coefficients on X. Note that
   it contains a field "\texttt{elike}" with the problem setup.  This may be useful to
   pass to "\texttt{elEval(...)}", etc.
 

\end{lstlisting}
\linesect{elModelSumm}
\begin{lstlisting}
  Print some basic information about the EL model solution "\texttt{res}"
 \end{lstlisting}
{\small\textbf{Usage:}}
\begin{lstlisting}
 
    "\texttt{[stderr,oidstat] = elModelSumm(res, names, [quiet])}"
 \end{lstlisting}
{\small\textbf{Inputs:}}
\begin{lstlisting}
 
    "\texttt{res}" is a solution object returned by elSolve(...)
    "\texttt{names}" is an optional vector of names for the elements of theta
    "\texttt{quiet}" is an optional parameter -- if it's true, printing is suppressed
 \end{lstlisting}
{\small\textbf{Outputs:}}
\begin{lstlisting}
 
     "\texttt{stderr}" is a column vector of computed standard errors.
     "\texttt{oidstat}" is the chi-squared statistic from an LR
         test of overidentifying restrictions.
 \end{lstlisting}
{\small\textbf{Details:}}
\begin{lstlisting}
 
     Reports the estimated theta from the model, standard errors,
     and a likelihood ratio-based test for the validity of the
     overidentifying restrictions.
 
     The standard errors are based on the normal asymptotic approximation
     to the estimator of theta, using a variance matrix based on the moments
     weighted by the computed "\texttt{pi}" values.
 

\end{lstlisting}
\linesect{elEval}
\begin{lstlisting}
  Evaluates the empirical likelihood function at the given point
 \end{lstlisting}
{\small\textbf{Usage:}}
\begin{lstlisting}
     res = elEval(elike, theta, p, lambda);
 
  "\texttt{elike}" should be a structure obtained from elSetup(...)
 
  "\texttt{method}" is 'EL','ET', or a CR constant parameter
 
  "\texttt{lambda}" contains the Lagrange multipliers, but may be omitted
           if you don't want the Hessian.
 
  "\texttt{res}" contains the following fields:
     "\texttt{res.f}"      --  objective function
     "\texttt{res.fgrad}"  --  gradient of objective
     "\texttt{res.H}"      --  Hessian of the Lagrangian (empty if lambda missing)
     "\texttt{res.c}"      --  constraints ( sum(pi(i) * m(i) )
     "\texttt{res.cJ}"     --  Jacobian of moments per observation
     "\texttt{res.mom}"    --  Moment conditions for each observation:
                        an (nObs x nMom) matrix with one column for each
                        moment condition and one row for each observation.
 

\end{lstlisting}
\linesect{elLRTest}
\begin{lstlisting}
  Computes an empirical likelihood ratio test of the hypotheses "\texttt{R*theta = b}"
 \end{lstlisting}
{\small\textbf{Usage:}}
\begin{lstlisting}
 
     "\texttt{testres = elLRTest(elres)}"
  
     "\texttt{testres = elLRTest(elres, R, b, joint)}"
 \end{lstlisting}
{\small\textbf{Inputs:}}
\begin{lstlisting}
 
     "\texttt{elres}" is a result structure returned by elSolve
 
     "\texttt{R}" is a K x nTheta matrix of linear hypotheses to test, while "\texttt{b}" is a
         K x 1 vector of values to test R against. If "\texttt{joint}" is
         omitted or false, then each row of the matrix is a separate
         linear hypothesis to be tested of the form:
 
             H0:  R(k,:) * theta = b
 
         If "\texttt{joint}" is true, then the vector hypothesis: H0: hyp * theta = b
         is tested.
 
         If "\texttt{R}" is omitted, by default elLRTest tests separately that each
         element of theta = 0. (So it is equivalent to "\texttt{hyp = eye(nTheta)}")
 \end{lstlisting}
{\small\textbf{Outputs:}}
\begin{lstlisting}
 
     "\texttt{pval}" is K x 1 vector for separate estimation or a scalar for joint
         estimation. It is the p-value for the LR test: the probability under
         the null that the LR statistic would be at least as large as the one
         obtained. It comes from the CDF of a chi-squared(dof) distribution,
         where "\texttt{dof}" is the rank of the hypothesis. If "\texttt{joint}" is false,
         separate p-values are computed for each hypothesis.
 
     "\texttt{lrstat}" the LR statistic obtained from each hypothesis:
 
         LR = -2 * ( L(theta_restricted) - L(theta_unres) )
 
  Notes:
 
     Each test requires re-solving the model, so it may take some time for a
     large model.
 

\end{lstlisting}
\linesect{elConfRegion}
\begin{lstlisting}
  Computes an EL likelihood-ratio confidence interval and possibly plots it.
  Currently only supports univariate confidence intervals.
 \end{lstlisting}
{\small\textbf{Primary Usage:}}
\begin{lstlisting}
 
     "\texttt{elConfRegion(elres, [whichVars], [confVals], [doPlot], [plotOpt])}"
 
         Arguments in [ ] are optional.
 
 \end{lstlisting}
{\small\textbf{Alternative Usage:}}
\begin{lstlisting}
 
     "\texttt{elConfRegion(elres, whichVars, confVals, 'precise')}"
 
         Slightly more accurate, but much slower, optimization-based means of
         finding confidence regions.
 \end{lstlisting}
{\small\textbf{Inputs:}}
\begin{lstlisting}
 
     "\texttt{elres}" is a result structure previously returned by elSolve(...)
 
     "\texttt{whichVars}" specifies the indices of theta for which confidence regions
         are desired. Leave as '[]' or omit to compute CRs for all indices.
 
     "\texttt{confVals}" specifies one or more confidence levels (e.g. [0.95,0.99]) to
         consider. Defaults to 0.95
 
     "\texttt{doPlot}" specifies that the profile p-value should be plotted
 
     "\texttt{plotOpt}" is an optional structure with fields specifying plotting
         options. Options include "\texttt{plotOpt.numPoints}" (number of points to sample),
         "\texttt{plotOpt.varnames}" (a cell array of variable names), "\texttt{plotOpt.xciLabel}"
         (whether to label the confidence interval on the x-axis), and
         "\texttt{plotOpt.horizCR}" (whether to draw a horizontal line for the
         confidence interval).
 

\end{lstlisting}
\linesect{elValues}
\begin{lstlisting}
  Computes the profile empirical likelihood, p-value, or LR test statistic at
  various points. Useful for plotting profile likelihoods.
 
  The LR stat corresponds to the null hypothesis that:
 
        "\texttt{theta(fixvars) = thetaList(fixvars,i)}"
 
  The test has "\texttt{length(fixvars)}" degrees of freedom unless otherwise specified.
 \end{lstlisting}
{\small\textbf{Usage:}}
\begin{lstlisting}
 
    "\texttt{[ll,lr,pv] = elValues(elres, thetaList, fixvars, [dof])}"
 \end{lstlisting}
{\small\textbf{Inputs:}}
\begin{lstlisting}
 
    "\texttt{elres}" is a result structure previously returned by elSolve(...)
            solving the unconstrained problem.
 
    "\texttt{thetaList}" is a matrix of size nFixed x nProfiles, where nFixed is the
            length of "\texttt{fixvars}" and nProfiles is the number of points at
            which to compute the profile likelihood.  Each column is a single
            value of the parameter vector at which to compute the
            likelihood.
 
    "\texttt{fixvars}" is an integer vector of length nFixed indicating which elements
            of theta should be held fixed. If fixvars is empty, it indicates
            that all elements of theta are fixed (only the p's are
            free). These all-theta-fixed problems can be solved faster.
 
     "\texttt{dof}" specifies the number of degrees of freedom to use when computing
         the p-values. (Only needed if p-values are computed)
 \end{lstlisting}
{\small\textbf{Outputs:}}
\begin{lstlisting}
 
     "\texttt{ll}" is a row vector containing the profile empirical log-likelihood at each
           point in thetaList. (This is the log likelihood obtained after
           maximizing over "\texttt{p}" and the theta elements not included in "\texttt{fixvars}",
           but holding the other theta elements fixed.)
 
     "\texttt{lr}" contains the profile empirical likelihood ratios (comparing the
          restricted and unrestricted models)
 
     "\texttt{pv}" contains the profile empirical p-value comparing the restricted and
          unrestricted models(based on the chi-squared approximation to the
          distribution of the LR stat)
 

\end{lstlisting}
